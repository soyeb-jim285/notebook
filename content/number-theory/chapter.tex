\chapter{Number theory}

\section{Modular arithmetic}
	\kactlimport{ModularArithmetic.h}
	\kactlimport{ModInverse.h}
	% \kactlimport{ModPow.h}
	\kactlimport{ModLog.h}
	% \kactlimport{ModSum.h}
	% \kactlimport{ModMulLL.h}
	\kactlimport{ModSqrt.h}

\section{Primality}
	\kactlimport{FastEratosthenes.h}
	\kactlimport{MillerRabin.h}
	\kactlimport{Factor.h}

\section{Divisibility}
	\kactlimport{euclid.h}
	% \kactlimport{Euclid.java}
  \subsection{Chinese Remainder Theorem}
  Let $m = m_1 \cdot m_2 \cdots m_k$, where $m_i$ are pairwise coprime. In addition to $m_i$, we are also given a system of congruences

$$\left\{\begin{array}{rcl}
    a & \equiv & a_1 \pmod{m_1} \\
    a & \equiv & a_2 \pmod{m_2} \\
      & \vdots & \\
    a & \equiv & a_k \pmod{m_k}
\end{array}\right.$$

where $a_i$ are some given constants. CRT will give the unique solution modulo $m$.
	\kactlimport{CRT.h}
  \kactlimport{CRT2.h}

	\subsection{Bézout's identity}
	For $a \neq $, $b \neq 0$, then $d=gcd(a,b)$ is the smallest positive integer for which there are integer solutions to
	$$ax+by=d$$
	If $(x,y)$ is one solution, then all solutions are given by
	$$\left(x+\frac{kb}{\gcd(a,b)}, y-\frac{ka}{\gcd(a,b)}\right), \quad k\in\mathbb{Z}$$
  \kactlimport{Diophantine.h}
	\kactlimport{phiFunction.h}

% \section{Fractions}
% 	\kactlimport{ContinuedFractions.h}
% 	\kactlimport{FracBinarySearch.h}

\section{Pythagorean Triples}
 The Pythagorean triples are uniquely generated by
 \[ a=k\cdot (m^{2}-n^{2}),\ \,b=k\cdot (2mn),\ \,c=k\cdot (m^{2}+n^{2}), \]
 with $m > n > 0$, $k > 0$, $m \bot n$, and either $m$ or $n$ even.

\section{Primes}
	$p=962592769$ is such that $2^{21} \mid p-1$, which may be useful. For hashing
	use 970592641 (31-bit number), 31443539979727 (45-bit), 3006703054056749
	(52-bit). There are 78498 primes less than 1\,000\,000.

	Primitive roots exist modulo any prime power $p^a$, except for $p = 2, a > 2$, and there are $\phi(\phi(p^a))$ many.
	For $p = 2, a > 2$, the group $\mathbb Z_{2^a}^\times$ is instead isomorphic to $\mathbb Z_2 \times \mathbb Z_{2^{a-2}}$.
\section{Fibonacchi}
Fibonacci numbers are defined by $F_0 = 0, F_1 = 1, F_n = F_{n-1} + F_{n-2}$. Again, $F_n = \frac{\phi^n - \hat{\phi}^n}{\sqrt{5}} \approx \frac{\phi^n}{\sqrt{5}} $, where $\phi = \frac{1 + \sqrt{5}}{2}$ and $\hat{\phi} = \frac{1 - \sqrt{5}}{2}$.
Some important properties of Fibonacci numbers:
\begin{align*}
  F_{n-1}F_{n+1} - F_n^2 = (-1)^n &&& F_{n+k} = F_{k-1}F_{n} + F_kF_{n+1} \\
  F_{2n} = F_n(F_{n-1} + F_{n+1}) &&& F_{2n+1} = F_n^2 + F_{n+1}^2 \\
  n | m \Leftrightarrow F_n | F_m &&& \gcd(F_m, F_n) = F_{\gcd(m,n)} \\
  F_A F_B=F_{k+1} F_A^2+F_k F_A F_{A-1} &&& \sum_{i=0}^n F_i^2=F_{n+1} F_n \\
  \sum_{i=0}^n F_i F_{i+1}=F_{n+1}^2-(-1)^n &&&  \sum_{i=0}^n F_i F_{i-1}=\sum_{i=0}^{n-1} F_i F_{i+1}\\
  \sum_{0 \leq k \leq n}\left(\begin{array}{c}
  n-k \\
  k
  \end{array}\right)=F_{n+1} &&& \\ 
  \operatorname{gcd}\left(F_n, F_{n+1}\right)=\operatorname{gcd}\left(F_n, F_{n+2}\right)&&&= \operatorname{gcd}\left(F_{n+1}, F_{n+2}\right)=1
\end{align*}
\kactlimport{Fibonacchi.h}
\section{Primitive Roots}

In modular arithmetic, a number $g$ is called a `primitive root modulo n` if every number coprime to $n$ is congruent to a power of $g$ modulo $n$. Mathematically, $g$ is a `primitive root modulo n` if and only if for any integer $a$ such that $\gcd(a, n) = 1$, there exists an integer $k$ such that:

$g^k \equiv a \pmod n$.

$k$ is then called the `index` or `discrete logarithm` of $a$ to the base $g$ modulo $n$. $g$ is also called the `generator` of the multiplicative group of integers modulo $n$.

In particular, for the case where $n$ is a prime, the powers of primitive root runs through all numbers from $1$ to $n-1$.
\kactlimport{PrimitiveRoot.h}
\section{Estimates}
	$\sum_{d|n} d = O(n \log \log n)$.

	The number of divisors of $n$ is at most around 100 for $n < 5e4$, 500 for $n < 1e7$, 2000 for $n < 1e10$, 200\,000 for $n < 1e19$.

\section{Mobius Function}
\[
	\mu(n) = \begin{cases} 0 & n \textrm{ is not square free}\\ 1 & n \textrm{ has even number of prime factors}\\ -1 & n \textrm{ has odd number of prime factors}\\\end{cases}
\]
  Mobius Inversion:
  \[ g(n) = \sum_{d|n} f(d) \Leftrightarrow f(n) = \sum_{d|n} \mu(d)g(n/d) \]
  Other useful formulas/forms:

  $ \sum_{d | n} \mu(d) = [ n = 1] $ (very useful)

  $ g(n) = \sum_{n|d} f(d) \Leftrightarrow f(n) = \sum_{n|d} \mu(d/n)g(d)$

\subsection{GCD and LCM}
if $m$ is any integer, then $\displaystyle \gcd(a + m {\cdot} b, b) = \gcd(a, b)$ \\
The gcd is a multiplicative function in the following sense: if $\displaystyle a_1$ and $\displaystyle a_2$ are relatively prime, then $\displaystyle \gcd(a_1 \cdot a_2, b) = \gcd(a_1, b) \cdot \gcd(a_2,b )$. \\
$\displaystyle \gcd(a, \operatorname{lcm}(b, c)) = \operatorname{lcm}(\gcd(a, b), \gcd(a, c))$. \\
$\displaystyle \operatorname{lcm}(a, \gcd(b, c)) = \gcd(\operatorname{lcm}(a, b), \operatorname{lcm}(a, c))$. \\
For non-negative integers $\displaystyle a$ and $b$, where $\displaystyle a$ and $b$ are not both zero, $\displaystyle \gcd({n^a} - 1, {n^b} - 1) = n^{\gcd(a,b)} - 1$ \\
$\displaystyle \gcd(a, b) = \displaystyle \sum_{k|a \, \text{and} \, k|b} {\phi(k)}$ \\
$\displaystyle \displaystyle \sum_{i=1}^n [\gcd(i, n) = k] = { \phi{\left(\frac{n}{k}\right)}}$ \\
$\displaystyle \displaystyle \sum_{k=1}^n \gcd(k, n) = \displaystyle \sum_{d|n} d \cdot {\phi{\left(\frac{n}{d}\right)}}$ \\
$\displaystyle \displaystyle \sum_{k=1}^n x^{\gcd(k,n)} = \displaystyle \sum_{d|n} x^d \cdot {\phi{\left(\frac{n}{d}\right)}}$ \\
$\displaystyle \displaystyle \sum_{k=1}^n \frac{1}{\gcd(k, n)} = \displaystyle \sum_{d|n} \frac{1}{d} \cdot {\phi{\left(\frac{n}{d}\right)}} = \frac{1}{n} \displaystyle \sum_{d|n} d \cdot \phi(d)$ \\
$\displaystyle \displaystyle \sum_{k=1}^n \frac{k}{\gcd(k, n)} = \frac{n}{2} \cdot \displaystyle \sum_{d|n} \frac{1}{d} \cdot {\phi{\left(\frac{n}{d}\right)}} = \frac{n}{2} \cdot \frac{1}{n} \cdot \displaystyle \sum_{d|n} d \cdot \phi(d)$ \\
$\displaystyle \displaystyle \sum_{k=1}^n \frac{n}{\gcd(k, n)} = 2 * \displaystyle \sum_{k=1}^n \frac{k}{\gcd(k, n)} - 1$, for $n > 1$ \\
$\displaystyle \displaystyle \sum_{i=1}^n \sum_{j=1}^n [\gcd(i, j) = 1] = \displaystyle \sum_{d=1}^n \mu(d) \lfloor {\frac{n}{d} \rfloor}^2$ \\
$\displaystyle \displaystyle \sum_{i=1}^n \displaystyle\sum_{j=1}^n \gcd(i, j) = \displaystyle \sum_{d=1}^n \phi(d) \lfloor {\frac{n}{d} \rfloor}^2$ \\
$\displaystyle \sum_{i=1}^n \sum_{j=1}^n i \cdot j[\gcd(i, j) = 1] = \sum_{i=1}^n \phi(i)i^2$ \\
$\displaystyle F(n) = \displaystyle \sum_{i=1}^n \displaystyle \sum_{j=1}^n \operatorname{lcm}(i, j) = \displaystyle \sum_{l=1}^n {\left(\frac{\left( 1 + \lfloor{\frac{n}{l} \rfloor} \right) \left( \lfloor{\frac{n}{l} \rfloor} \right)} {2} \right)}^2 \displaystyle \sum_{d|l} \mu(d)ld$ \\ 

 $ g(n) = \sum_{1 \leq m \leq n} f(\left\lfloor\frac{n}{m}\right \rfloor ) \Leftrightarrow f(n) = \sum_{1\leq m\leq n} \mu(m)g(\left\lfloor\frac{n}{m}\right\rfloor)$
