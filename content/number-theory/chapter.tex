\chapter{Number theory}

\section{Modular arithmetic}
	\kactlimport{ModularArithmetic.h}
	\kactlimport{ModInverse.h}
	\kactlimport{ModPow.h}
	\kactlimport{ModLog.h}
	\kactlimport{ModSum.h}
	\kactlimport{ModMulLL.h}
	\kactlimport{ModSqrt.h}

\section{Primality}
	\kactlimport{FastEratosthenes.h}
	\kactlimport{MillerRabin.h}
	\kactlimport{Factor.h}

\section{Divisibility}
	\kactlimport{euclid.h}
	% \kactlimport{Euclid.java}
  \subsection{Chinese Remainder Theorem}
  Let $m = m_1 \cdot m_2 \cdots m_k$, where $m_i$ are pairwise coprime. In addition to $m_i$, we are also given a system of congruences

$$\left\{\begin{array}{rcl}
    a & \equiv & a_1 \pmod{m_1} \\
    a & \equiv & a_2 \pmod{m_2} \\
      & \vdots & \\
    a & \equiv & a_k \pmod{m_k}
\end{array}\right.$$

where $a_i$ are some given constants. CRT will give the unique solution modulo $m$.
	\kactlimport{CRT.h}
  \kactlimport{CRT2.h}

	\subsection{Bézout's identity}
	For $a \neq $, $b \neq 0$, then $d=gcd(a,b)$ is the smallest positive integer for which there are integer solutions to
	$$ax+by=d$$
	If $(x,y)$ is one solution, then all solutions are given by
	$$\left(x+\frac{kb}{\gcd(a,b)}, y-\frac{ka}{\gcd(a,b)}\right), \quad k\in\mathbb{Z}$$
  \kactlimport{Diophantine.h}
	\kactlimport{phiFunction.h}

\section{Fractions}
	\kactlimport{ContinuedFractions.h}
	\kactlimport{FracBinarySearch.h}

\section{Pythagorean Triples}
 The Pythagorean triples are uniquely generated by
 \[ a=k\cdot (m^{2}-n^{2}),\ \,b=k\cdot (2mn),\ \,c=k\cdot (m^{2}+n^{2}), \]
 with $m > n > 0$, $k > 0$, $m \bot n$, and either $m$ or $n$ even.

\section{Primes}
	$p=962592769$ is such that $2^{21} \mid p-1$, which may be useful. For hashing
	use 970592641 (31-bit number), 31443539979727 (45-bit), 3006703054056749
	(52-bit). There are 78498 primes less than 1\,000\,000.

	Primitive roots exist modulo any prime power $p^a$, except for $p = 2, a > 2$, and there are $\phi(\phi(p^a))$ many.
	For $p = 2, a > 2$, the group $\mathbb Z_{2^a}^\times$ is instead isomorphic to $\mathbb Z_2 \times \mathbb Z_{2^{a-2}}$.
\section{Fibonacci}
Fibonacci numbers are defined by $F_0 = 0, F_1 = 1, F_n = F_{n-1} + F_{n-2}$. Again, $F_n = \frac{\phi^n - \hat{\phi}^n}{\sqrt{5}} \approx \frac{\phi^n}{\sqrt{5}} $, where $\phi = \frac{1 + \sqrt{5}}{2}$ and $\hat{\phi} = \frac{1 - \sqrt{5}}{2}$.
Some important properties of Fibonacci numbers:
\begin{align*}
  F_{n-1}F_{n+1} - F_n^2 = (-1)^n &&& F_{n+k} = F_{k-1}F_{n} + F_kF_{n+1} \\
  F_{2n} = F_n(F_{n-1} + F_{n+1}) &&& F_{2n+1} = F_n^2 + F_{n+1}^2 \\
  n | m \Leftrightarrow F_n | F_m &&& \gcd(F_m, F_n) = F_{\gcd(m,n)} \\
\end{align*}
\kactlimport{Fibonacci.h}
\section{Estimates}
	$\sum_{d|n} d = O(n \log \log n)$.

	The number of divisors of $n$ is at most around 100 for $n < 5e4$, 500 for $n < 1e7$, 2000 for $n < 1e10$, 200\,000 for $n < 1e19$.

\section{Mobius Function}
\[
	\mu(n) = \begin{cases} 0 & n \textrm{ is not square free}\\ 1 & n \textrm{ has even number of prime factors}\\ -1 & n \textrm{ has odd number of prime factors}\\\end{cases}
\]
  Mobius Inversion:
  \[ g(n) = \sum_{d|n} f(d) \Leftrightarrow f(n) = \sum_{d|n} \mu(d)g(n/d) \]
  Other useful formulas/forms:

  $ \sum_{d | n} \mu(d) = [ n = 1] $ (very useful)

  $ g(n) = \sum_{n|d} f(d) \Leftrightarrow f(n) = \sum_{n|d} \mu(d/n)g(d)$

 $ g(n) = \sum_{1 \leq m \leq n} f(\left\lfloor\frac{n}{m}\right \rfloor ) \Leftrightarrow f(n) = \sum_{1\leq m\leq n} \mu(m)g(\left\lfloor\frac{n}{m}\right\rfloor)$
